\documentclass{mwrep}
\usepackage[utf8]{inputenc}
\usepackage{polski}
\usepackage{amsmath}

\title{Podpisy biometryczne na tablecie \\ i ich porównanie z podpisami na papierze\\ 
Raport 6}
\author{Mikołaj Balcerek, Bartosz Hejduk, Mieczysław Krawiarz, \and Adam Kulczycki, Mikołaj Pabiszczak, Michał Szczepanowski, \and Dawid Twardowski, Adrianna Załęska}
\date{12 września 2017}

\begin{document}
\maketitle
{\let\clearpage\relax 
\chapter{Obecne osoby}}
\begin{enumerate}
    \item Mikołaj Balcerek
    \item Bartosz Hejduk
    \item Mieczysław Krawiarz
    \item Adam Kulczycki
    \item Mikołaj Pabiszczak
    \item Michał Szczepanowski
    \item Dawid Twardowski
    \item Adrianna Załęska
\end{enumerate}


{\let\clearpage\relax 
\chapter{Zadania zaplanowane na dzień 12 września}}
\begin{enumerate}
	\item Spotkanie z przedstawicielami firmy IC Solutions.
	\item Praca nad generowaniem wykresów obrazujących zmiany siły nacisku oraz prędkości / przyspieszenia pióra. Pomoże nam to w szukaniu zależności pomiędzy złożonymi parafkami i eksperymentalnego sprawdzania naszych hipotez dot. ich weryfikacji.
	\item Normalizacja początku pozycji podpisu. Złożony podpis powinien być odpowiednio przycinany, by jego początek znajdował się w zbliżonym punkcie w porównaniu do wszystkich innych sygnatur.
	\item Integracja i testowanie funkcjonalności z poprzednich dni: generowanie poglądu .gif obrazu, badanie przyspieszeń i szybkości pociągnięć, profile użytkowników.
\end{enumerate}


{\let\clearpage\relax 
\chapter{Zrealizowane zadania}}
\begin{enumerate}
    \item Rozmowa z przedstawicielami IC Solutions - omówienie postępów i planu dalszej pracy. 
    Naszym celem powinno być jak najszybsze dojście do fazy eksperymentalnej, by móc zweryfikować nasze hipotezy dotyczące istotności poszczególnych zmiennych biometrycznych.
    \item Tymczasowa rezygnacja z graficznej analizy sygnatur. 
    Po rozmowie z przedstawicielami firmy IC Solutions stwierdziliśmy, że analiza pozagraficzna powinna być głównym celem naszego projektu. Implementacja analizy obrazów byłaby wyjątkowo czasochłonna i wymagająca. Rozpatrzymy skorzystanie z gotowych zewnętrznych bibliotek opartych na niewirusowych licencjach na późniejszym etapie projektu.
    \item Przypisywanie autora do podpisu. W programie możliwe jest teraz określenie do kogo należy złożony podpisów. Ta funkcjonalność została zintegrowana i przetestowana.
	\item Zapisywanie poglądu podpisów graficznie do .gif
	\item Klasa derivatives licząca pochodne (szybkość i przyspieszenia w kierunku X, Y, sumaryczna)
	\item Zapis bogatszych informacji o pociągnięciach - czas oraz grubość
	\end{enumerate}

{\let\clearpage\relax \chapter{Zadania na najbliższe dni}}
\begin{enumerate} 
    \item Ponowna próba zapisywania danych o kącie nachylenia długopisu względem ekranu. Za pierwszą próbą API firmy Microsoft zwracał stały kąt 0, co zniechęciło nas do pracy nad tym czynnikiem. Przedstawiciele IC Solutions wskazują, że może być to ważna funkcjonalność.
    \item Wprowadzenie stałej wielkości pola podpisu (90 x 27 mm). W trakcie spotkania został nam przedstawiony najczęstszy use-case dla podpisu na przykładzie formularza. Dzięki temu możemy uprościć nasze podejście i założyć, że sygnatura zmieści się w polu 90 x 27 mm.
    \item Algorytm skalujący podpis z użyciem średniej i odchyleń standardowych. Waga tego zadania zmniejszyła się ze względu na nieskalowalny charakter naszego pola do podpis (patrz punkt poprzedni)
    \item Dalsza praca nad: normalizacją początków sygnatur, wykresami i prędkościami pociągnięć.
    \item Wyłączenie gumki (możliwe jest zmazanie pociągnięć za pomocą końca rysika Surface Pen)
    \item Połączenie metod weryfikacji w celu utworzenia tzw. Trustworthiness score. Będzie to wynik przybierający wartości 1-100 określający naszą pewność w autentyczność podpisu. Należy również wziąć uwagę sztucznie idealne podpisy (idealnie skopiowane lub prawie) i je odrzucać.
    To zadanie będzie składać się z krótkiej pracy programistycznej oraz prawdopodobnie bardzo długiego iterowania i zmieniania wag poszczególnych metod badania autentyczności.
    \item Stworzenie lokalnej bazy danych i połączenie jej z naszą aplikacją w celu przechowywania podpisów i informacji o nich.   
    \item Prowadzanie eksperymentów w celu sprawdzenia słuszności wszystkich naszych poprzednich założeń na gronie innych uczestników Poznańskich Praktych Badawczych.
\end{enumerate}



\end{document}