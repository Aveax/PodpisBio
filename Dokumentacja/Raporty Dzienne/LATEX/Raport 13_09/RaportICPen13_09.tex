\documentclass{mwrep}
\usepackage[utf8]{inputenc}
\usepackage{polski}
\usepackage{amsmath}

\title{Podpisy biometryczne na tablecie \\ i ich porównanie z podpisami na papierze\\ 
Raport 7}
\author{Mikołaj Balcerek, Bartosz Hejduk, Mieczysław Krawiarz, \and Adam Kulczycki, Mikołaj Pabiszczak, Michał Szczepanowski, \and Dawid Twardowski, Adrianna Załęska}
\date{12 września 2017}

\begin{document}
\maketitle
{\let\clearpage\relax 
\chapter{Obecne osoby}}
\begin{enumerate}
    \item Mikołaj Balcerek
    \item Bartosz Hejduk
    \item Mieczysław Krawiarz
    \item Adam Kulczycki
    \item Mikołaj Pabiszczak
    \item Michał Szczepanowski
    \item Dawid Twardowski
    \item Adrianna Załęska
\end{enumerate}


{\let\clearpage\relax 
\chapter{Zadania zaplanowane na dzień 13 września}}
\begin{enumerate}
	\item Zbieranie informacji na temat algorytmu Dynamic Time Wrapping i jego implementacji.
	\item Wybieranie technologii do stworzenia bazy danych.
	\item Tworzenie zalążków warstwy biznesowej (łączenie się z firmą IC Solutions).
	\item Szukanie informacji o ukrytych łańcucach Markowa.
	\item Zapoznanie się z wprowadzeniem do algotytmów Bauma-Welcha.
	\item Stworzenie klasy TimeSizeProbe do zbierania informacji o rozmiarach i czasach rzeczywistych (podczas przyciśnięcia rysika do ekranu) pisania pociągnięć z osobna.
\end{enumerate}


{\let\clearpage\relax 
\chapter{Zrealizowane zadania}}
\begin{enumerate}
	\item Odnalezenie klasy InkStroke przechowującej bogate informacje o pociągnięciach.
	\item Praca nad generowaniem wykresów obrazujących zmiany siły nacisku oraz prędkości / przyspieszenia pióra. Pomoże nam to w szukaniu zależności pomiędzy złożonymi parafkami i eksperymentalnego sprawdzania naszych hipotez dot. ich weryfikacji.
	\item Dalsza praca nad normalizacją początku pozycji podpisu.
	\item Stworzenie lokalnej bazy danych.
	\item Zbieranie infomacji na temat przydatnych algorytmów w weryfikacji zgodności podpisów. 
	\item Dalsze prace nad normalizacją podpisów. Ukończone skalowanie podpisu i przeniesienie do do punktu (0,0).
	\item Częściowa impemetacja stałej wielkości pola podpisu (90 x 27 mm).
	\end{enumerate}

{\let\clearpage\relax \chapter{Zadania na najbliższe dni}}
\begin{enumerate} 
    \item Ponowna próba zapisywania danych o kącie nachylenia długopisu względem ekranu. Za pierwszą próbą API firmy Microsoft zwracał stały kąt 0, co zniechęciło nas do pracy nad tym czynnikiem. Przedstawiciele IC Solutions wskazują, że może być to ważna funkcjonalność.
    \item Wyłączenie gumki (możliwe jest zmazanie pociągnięć za pomocą końca rysika Surface Pen)
    \item Połączenie metod weryfikacji w celu utworzenia tzw. Trustworthiness score. Będzie to wynik przybierający wartości 1-100 określający naszą pewność w autentyczność podpisu. Należy również wziąć uwagę sztucznie idealne podpisy (idealnie skopiowane lub prawie) i je odrzucać.
    To zadanie będzie składać się z krótkiej pracy programistycznej oraz prawdopodobnie bardzo długiego iterowania i zmieniania wag poszczególnych metod badania autentyczności.
    \item Stworzenie lokalnej bazy danych i połączenie jej z naszą aplikacją w celu przechowywania podpisów i informacji o nich.   
    \item Prowadzanie eksperymentów w celu sprawdzenia słuszności wszystkich naszych poprzednich założeń na gronie innych uczestników Poznańskich Praktych Badawczych.
    \item Zakończenie prac nad stałą wielkością pola.
\end{enumerate}



\end{document}