\documentclass{mwrep}
\usepackage[utf8]{inputenc}
\usepackage{polski}


\title{Podpisy biometryczne na tablecie \\ i ich porównanie z podpisami na papierze\\ 
Raport 3}
\author{Mikołaj Balcerek, Bartosz Hejduk, Mieczysław Krawiarz, \and Adam Kulczycki, Mikołaj Pabiszczak, Michał Szczepanowski, \and Dawid Twardowski, Adrianna Załęska}
\date{7 września 2017}

\begin{document}
\maketitle
{\let\clearpage\relax 
\chapter{Obecne osoby}}
\begin{enumerate}
    \item Mikołaj Balcerek
    \item Bartosz Hejduk
    \item Mieczysław Krawiarz
    \item Adam Kulczycki
    \item Mikołaj Pabiszczak
    \item Michał Szczepanowski
    \item Dawid Twardowski
    \item Adrianna Załęska
\end{enumerate}


{\let\clearpage\relax 
\chapter{Zadania zaplanowane na dziś}}
\begin{enumerate}
	\item Budowa systemu zbierania danych o podpisach biometrycznych.
	\item Zebranie różnych pomysłów dotyczących weryfikacji autentyczności podpisu - burza mózgów.
	\item Zbudowanie przykładowej bazy danych.
\end{enumerate}


{\let\clearpage\relax 
\chapter{Zrealizowane zadania}}
\begin{enumerate}
	\item Zgromadzenie pomysłów dotyczących weryfikacji autentyczności podpisu:
		\begin{itemize}
		\item Stosunek długości i szerokości podpisu.
		\item Liczba oderwań długopisu od ekranu.
		\item Stosunek wielkości podpisu do czasu jego składania.
		\item Badanie przebiegu funkcji siły nacisku w zależności od czasu. \\ (np. monotoniczność funkcji)
		\end{itemize}
	\item Dodanie kolejnych funkcjonalności do prototypu aplikacji.

\end{enumerate}

{\let\clearpage\relax \chapter{Zadania na najbliższe dni}}
\begin{enumerate}
    \item Zebranie jak największej liczby podpisów od różnych osób.
    \item Testowanie różnych modeli weryfikacji autentyczności podpisu.
\end{enumerate}
\end{document}